% Opcje klasy 'iithesis' opisane sa w komentarzach w pliku klasy. Za ich pomoca
% ustawia sie przede wszystkim jezyk i rodzaj (lic/inz/mgr) pracy, oraz czy na
% drugiej stronie pracy ma byc skladany wzor oswiadczenia o autorskim wykonaniu.
\documentclass[shortabstract,polish,lic]{iithesis}
\let\lll\relax
\usepackage{mathabx}
\usepackage[utf8]{inputenc}
\usepackage{fancyhdr}
\usepackage[toc]{appendix}
\usepackage{bookmark}
\usepackage{enumerate}
\usepackage{csvsimple}
\usepackage{float}

\usepackage{pdfpages}



\pagestyle{fancy}
\fancyhf{}
\fancyfoot[CO,CE]{\thepage}
\fancyhead[RE]{\leftmark}
\fancyhead[LO]{\rightmark}
\renewcommand{\chaptermark}[1]{\markboth{Paweł Dybiec, Instytut Informatyki UWr}{}}
\renewcommand{\sectionmark}[1]{\markright{Weryfikacja możliwości sterowania łazikiem za pomocą sieci neuronowych}}

\polishtitle{Weryfikacja możliwości sterowania łazikiem za pomocą sieci neuronowych}
\englishtitle{Veryfing the remote control capabilities of rover using neural networks}

\author{Paweł Dybiec}
\advisor{dr Marek Materzok}
\date{12 II 2018} % Data zlozenia pracy
\transcriptnum{271900} % Numer indeksu
\advisorgen{dr Marka Materzoka} % Nazwisko promotora w dopelniaczu

\polishabstract{Sieci neuronowe są w stanie kierować samochodem na podstawie obrazu z
kamery\cite{nvidia}. Tematem tej pracy implementacja i przetestowanie autonomicznej jazdy
łazika Aleph 1 korzystającej z konwolucyjnych sieci neuronowych.}

\englishabstract{TODO ENG abstract}


\begin{document}
\chapter{Preliminaria}
Ta praca została zrealizowana w ramach przedmiotu "Projekt: autonomiczna jazda łazikiem".
Z jego powodu(trzeba zmienic to wyrazenie), powstało wiele rozwiązań dla zadań z 
"konkursów łazikowych".

Żaden spośród łazików biorących udział w University Rover Challenge nie 
używa sieci neuronowych bezpośrednio do nawigacji , ale prawie wszystkie używają
ROS ( Robot Operating System ) jako podstawy całego oprogramowania. Z tego powodu
w tym rozdziale poruszone będą:
\begin{itemize}
  \item Podstawy sieci neuronowych.
  \item Architektura ROS
  \item Autonomia Aleph 1
\end{itemize}

\section{Podstawy sieci neuronowych}
\subsection{Jak działają}
\subsection{Jak trenować}
\subsection{Warstwy typowe dla CNN}
\subsection{Dlaczego działają}


\section{ROS}
Nakładka na ubuntu
\subsection{Master}
\subsection{Node}
\subsection{Gotowe moduły}
tf,kamery,konwersje obrazków/strumieni

\section{Autonomia Aleph 1}
Co zostało zrobione na przedmiocie:
\begin{itemize}
  \item Sprzęt (mnóstwo)
  \item Mapa 3d (RTAB\_MAP)
  \item Rozpoznawanie klawiatur/piłek tenisowych
  \item Symulator
  \item kilka sieci obraz->kierownica
  \item wrappery/konwertery różnych protokołów/formatów
\end{itemize}

\chapter{Sieć pod symulator}
W celu autonomicznej jazdy wytrenowałem konwolucyjną sieć neuronową (CNN)
przetwarzającą obraz z kamery bezpośrednio w porządaną prędkość liniową
oraz obrotową. Takie podejście pozwala szybko zbierać dane uczące, wystarczy
tylko nagrać obraz z kamery oraz prędkość nadaną przez kierowcę.
\begin{figure}[h]
  \centering
  \fbox{
  \scalebox{0.5}{\includegraphics*[viewport=0 1300 600 2200]{img/model.png}}
  }
\end{figure}
\begin{figure}
  \centering
  \fbox{
  \scalebox{0.5}{\includegraphics*[viewport=0 0000 600 1300]{img/model.png}}
  }
  \label{model}
  \caption{Architektura sieci}
\end{figure}
Wersja sterująca w symulatorze powstała, żeby odrzucić modele, które nie radzą
sobie w tak prostych warunkach. Dodatkowo zbieranie danych oraz testowanie
modelu jest łatwiejsze, ponieważ nie wymaga przygotowywania sprzętu, oraz
opuszczenie toru przez model jest nieszkodliwe w porównaniu do opuszczenia
drogi przez fizycznego łazika.

\section{Dlaczego taka (a nie mniejsza)}
W sieci pięciokrotnie pojawia się sekwencja warstwa konwolucyjna -> dropout 
całych warstw ->max pooling.
Celem poolingu jest zmniejszenie liczby parametrów oraz zapobieganie 
przetrenowaniu. Max pooling dzieli obraz na bloki ustalonego rozmiaru i 
dla każdego z nich wyznacza maksimum, w ten sposób rozmiar 'feature maps' 
wielokrotnie się zmniejsza.

Dlaczego tylko 1 dense

\section{Dane}
Jak długie przejazdy, i ile ich: 2 po 20 minut

Co gdyby zmniejszyć rozdzielczość ewaluowanych obrazkow do 16x8: jest ok

Jak wzbogacane: obrazy z 3 kamer + flip na środkowej


\chapter{Wyniki sieci}
Wytrenowana sieć potrafi przejechać zarówno cały tor na symulatorze, jak i
podziemny garaż instytutu. Na dodatek sieć trenowana pod symulator uczyła się
 jeździć tylko przeciwnie do ruchu wskazówek zegara. Po ustawieniu modelu w przeciwnym
kierunku, sieć potrafi bezproblemowo przejechać cały tor.

\section{Na co zwraca uwagę}
Aktywność sieci dla obrazków została wygenerowana za pomocą metody
Integrated Gradients\footnote{\href{https://arxiv.org/abs/1703.01365}{https://arxiv.org/abs/1703.01365}}.

Co było oczywiste w przypadku symulatora, sieć zwraca głównie uwagę na miejsca,
gdzie pojawiają się granice drogi \ref{sim_act}. Co ciekawe, reaguje też na ścianę
tworzącą horyzont, ponieważ zmienia wygląd w zależności od odległości i może
pomóc w orientacji (na trasie treningowej).
\begin{figure}
  \centering
  \fbox{
  \scalebox{0.5}{\includegraphics{img/sim_img.png}}
  }
  \caption{Obraz z symulatora}
  \label{sim_img}
\end{figure}
\begin{figure}
  \centering
  \fbox{
  \scalebox{0.5}{\includegraphics{img/sim_img_act.png}}
  }
  \caption{Na co sieć patrzy, symulator}
  \label{sim_act}
\end{figure}
\begin{figure}
  \centering
  \fbox{
  \scalebox{0.5}{\includegraphics{img/real_img.png}}
  }
  \caption{Obraz z nagrania}
  \label{real_img}
\end{figure}
\begin{figure}
  \centering
  \fbox{
    \scalebox{0.5}{\includegraphics{img/real_img_act.png}}
  }
  \caption{Na co sieć patrzy, nagranie}
  \label{real_act}
\end{figure}

Z kolei dla łazika intensywność w najbardziej aktywnym miejscu jest dużo mniejsza \ref{real_act}.
Oznacza to, że nie sugeruje się tylko jednym obszarem z kamery. Najbardziej jednak 
zwraca uwagę na kratkę na podłodze, która mogłaby wystarczyć do nawigacji.

\section{W porównaniu do nagrania}
Na wykresie \ref{plot_ang} widać, że sieć (pomarańczowy kolor) mniej gwałtownie 
zmienia szybkość obrotu niż kierowca (kolor niebieski). Jednak sieć reaguje w podobnych momentach co kierowca na konieczność wykonania skrętu.

\begin{figure}
  \centering
  \fbox{
    \scalebox{0.5}{\includegraphics{img/real_data_ang.png}}
  }
  \caption{Prędkość obrotowa: sieć vs kierowca}
  \label{plot_ang}
\end{figure}

\section{Wpływ architektury}
W przypadku sieci pod symulator, usunięcie niektórych warstw konwolucyjnych
pozwalało modelowi nadal utrzymywać się na torze, natomiast taka sama zredukowana architektura 
nie radziła sobie dobrze w przypadku nagrań z prawdziwego łazika. Z kolei
usunięcie nieliniowości z warstw konwolucyjnych ograniczyło zdolności sieci do takiego stopnia,
że nie potrafiła się utrzymać na wirtualnym torze.

Usunięcie dropoutu bardzo szybko powodowało overfitting i sieć radziła sobie
dobrze tylko na danych uczących. Z kolei dodanie warstw liniowych na końcu nie 
poprawiało, ani nie pogorszało zbytnio wydajności sieci, przynajmniej dla
nagrań z symulatora. Na tej podstawie można wywnioskować, że większość interesujących 
cech obrazu została już znaleziona w ramach warstw konwolucyjnych, więc dla tak prostych danych zwiększenie modelu jest nieefektywne.

Co ciekawe, w przypadku wytrenowanego już modelu do symulatora zredukowanie 
rozdzielczości obrazów dziesięciokrotnie w każdym wymiarze (z rozdzielczości 
320x160 do 32x16),
i zwykłe przeskalowanie w górę przed zewaluowaniem wystarczy, żeby utrzymać się 
na torze.

Ponadto, sieć uczona na obrazie kolorowym działa bezproblemowo, gdy
zredukuje się obraz do skali szarości. Jedyne, co należy wykonać to stworzyć obraz kolorowy, w którym każdy z kanałów RGB będzie powtórzonym obrazem wejściowym.


\chapter{Co dalej}
RNN - sam wyciągnie kontekst

Na wersji sim-only - funkcja kosztu w zależności od odległości od trasy, może nagradzać szybkie przejazdy bo inaczej będzie stać w miejscu
Da się podciągnąć dla prawdziwej ale trzeba by jakoś użyć odo.

Reinforced learning - kara za każdą interwencję (może nie 0-1 tylko proporcjonalna od
róznicy outputów)



\begin{thebibliography}{1}

\bibitem{nvidia} Mariusz Bojarski,
               Davide Del Testa,
               Daniel Dworakowski,
               Bernhard Firner,
               Beat Flepp,
               Prasoon Goyal,
               Lawrence D. Jackel,
               Mathew Monfort,
               Urs Muller,
               Jiakai Zhang,
               Xin Zhang,
               Jake Zhao,
               Karol Zieba \href{https://arxiv.org/abs/1604.07316}{End to End Learning for Self-Driving Cars}, 2016
\bibitem{cnn} Alex Krizhevsky, Ilya Sutskever,Geoffrey E. Hinton  \href{https://www.cs.toronto.edu/~kriz/imagenet_classification_with_deep_convolutional.pdf}{ImageNet Classification with Deep Convolutional Neural Networks}, 2012
\bibitem{ig} Mukund Sundararajan, Ankur Taly, Qiqi Yan  \href{https://arxiv.org/abs/1703.01365}{Axiomatic Attribution for Deep Networks}, 2017

\end{thebibliography}
\renewcommand\appendixtocname{Dodatki}
\bookmarksetupnext{level=part}

\end{document}
