\chapter{Preliminaria}
Ta praca została zrealizowana w ramach przedmiotu "Projekt: autonomiczna jazda łazikiem".
Z jego powodu(trzeba zmienic to wyrazenie), powstało wiele rozwiązań dla zadań z 
"konkursów łazikowych".

Żaden spośród łazików biorących udział w University Rover Challenge nie 
używa sieci neuronowych bezpośrednio do nawigacji , ale prawie wszystkie używają
ROS ( Robot Operating System ) jako podstawy całego oprogramowania. Z tego powodu
w tym rozdziale poruszone będą:
\begin{itemize}
  \item Podstawy sieci neuronowych.
  \item Architektura ROS
  \item Autonomia Aleph 1
\end{itemize}

\section{Podstawy sieci neuronowych}
\subsection{Jak działają}
\subsection{Jak trenować}
\subsection{Warstwy typowe dla CNN}
\subsection{Dlaczego działają}


\section{ROS}
Nakładka na ubuntu
\subsection{Master}
\subsection{Node}
\subsection{Gotowe moduły}
tf,kamery,konwersje obrazków/strumieni

\section{Autonomia Aleph 1}
Co zostało zrobione na przedmiocie:
\begin{itemize}
  \item Sprzęt (mnóstwo)
  \item Mapa 3d (RTAB\_MAP)
  \item Rozpoznawanie klawiatur/piłek tenisowych
  \item Symulator
  \item kilka sieci obraz->kierownica
  \item wrappery/konwertery różnych protokołów/formatów
\end{itemize}
