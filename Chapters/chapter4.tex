\chapter{Co dalej}
Najprostszym następnym krokiem jest zwiększenie danych o dodatkowy wymiar, i nauczenie takiej
sieci decyzji na podstawie $k$ (niekoniecznie) ostatnich zdjęć. Innym prostym rozwiązaniem,
które można z tym połączyć jest zmiana perspektywy kamery na zdjęcie z góry.

Bardziej ambitnym pomysłem jest wytrenowanie rekurencyjnej sieci neuronowej (RNN),
gdyby ją dobrze nauczyć sama wyciągnie kontekst. Ale problemem przy jej trenowaniu
będzie fakt, że prostą strategią dla takiej sieci jest powtarzanie ostatniego wypisanego
wyniku, a to dlatego że prędkość jest ciągła.

Kolejnym rozwiązaniem jest reinforced learning, sieć karało by się za
każdą interwencję lub wyjechanie poza trasę. Niestety problemem tutaj jest 
fakt, że jak błąd prawdziwego pojazdu może być kosztowny lub niebezpieczny.

Oczywiście pozostają też rozwiązania nie używające sieci neuronowych, można
przykładowo stworzyć program pilnujący aby łazik nie wjechał w przeszkodę.


